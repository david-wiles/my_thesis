\chapter{Security Scanning and Remediation Patterns}

The tasks created to scan and fix security issues in software can be reduced to a number of similar patterns. Actions
used to scan for software issues can be reduced to the following groups:

\begin{center}
    \begin{tabular}{||p{0.3\textwidth} p{0.3\textwidth} p{0.3\textwidth}||}
        \hline
        Action & Can Be Automated & Needs Only Unix Tools  \\ [0.5ex]
        \hline\hline
        Software CLI interaction & Yes & Yes \\
        \hline
        Configuration File Search & Yes & Depends on Software \\
        \hline
        XML/JSON/YAML File Parse & Yes & No \\
        \hline
        Unix Tools Action & Yes & Yes \\ [0.5ex]
        \hline
    \end{tabular}
\end{center}

Likewise, issue remediation can be reduced into the following groups:

\begin{center}
    \begin{tabular}{||p{0.3\textwidth} p{0.3\textwidth} p{0.3\textwidth}||}
        \hline
        Action & Can Be Automated & Needs Only Unix Tools  \\ [0.5ex]
        \hline\hline
        Software CLI Action & Yes & No \\
        \hline
        Configuration File Find and Replace & Yes & Yes \\
        \hline
        Manual Remediation & No & No \\
        \hline
        XML/JSON/YAML File Update & Yes & No \\ [0.5ex]
        \hline
    \end{tabular}
\end{center}

Note that to fix a specific issue some automated tasks may break the system so a manual action would be preferable.


\section{Scanning Techniques}

Software configuration scanning techniques will be defined in the following section along with broad examples of how
to implement the technique.

\subsection{Interaction with a Software's CLI}

Some software will be primarily interacted with through a command-line interface. An example of this is Docker. Although
all actions can be performed through an HTTP interface to the Docker daemon, the most common and well-documented way to
interact is with the Docker command-line client. This client is maintained by Docker itself and is intended to be the
primary way that the user manages Docker containers and the Docker daemon.

Software such as this will provide commands that can be used to determine how the software is set up. There will likely
be ways to view configuration and modify configuration from the command line. With Docker, it is possible to inspect a
container to view all the information related to it, for example. This output can then be checked to ensure that certain
policies are in place.

\subsection{Configuration File Parsing}

One common way to configure software in Linux is through configuration files. These files can define the
behavior of a program and so parsing these files can help us determine what that behavior will be and whether it
matches our requirements.

These configuration files could be as simple as a key-value store as seen in the SSH configuration file. It is relatively simple
to use Unix Tools alone to determine whether our ssh configuration meets our requirements since a Regex search would
reliably determine whether a certain parameter is enabled. For example, if we want to disable password authentication,
we would search for a line such as

\begin{lstlisting}[style=AMMA]
PasswordAuthentication yes
\end{lstlisting}

This can be done with a regular expression.

Other software may use a DSL to define the configuration. These programs may have behavior too complex to define with a
file as simple as a key-value store and would require extra syntax to concisely define its behavior. The webserver
Nginx is a good example of software with a DSL used in configuration files. The .conf files may be easy for the user
to read and understand, but a computer would be unable to track the context of the file without using special software
to parse the file. This makes scanning configuration files with a DSL more similar to parsing XML files since it is
often not possible to reliably determine whether a configuration is correct using only Unix Tools.

\subsection{XML/JSON/YAML File Parsing}

XML, JSON, and YAML are popular markup languages. Due to their widespread use and simple syntax to understand, many
software projects may choose to adopt one of them as their configuration file's syntax instead of creating a DSL. This
standardization would make it much easier for other programs to interpret the files since parsers would often be
included in the language's standard library.

Despite this, scanning a configuration file that uses a markup language would not be possible without help from an
external tool. This tool could be generic, like jq, or domain-specific for the particular software. Either way, the
file would need to be parsed and checked against a list of requirements to determine if the setup meets standards.

\subsection{Actions With Unix Tools Only}

Setting up a new program also involves configuring the system it will run on. There must be a user to run the process
and that process may require AppArmor policies to ensure that the process can't damage other parts of the system.
Scanning these parts of the Linux kernel would only require built-in tools.

\section{Remediation Techniques}

Software configuration remediation techniques will be defined in the following section along with broad examples of how
to implement the technique.

\subsection{Interaction with a Software's CLI}

Many software projects will provide a CLI which allows for direct modification of its behavior. If we have identified
a specific issue we may be able to use this CLI to modify the program's behavior without needing to restart the program.
This CLI update would be a relatively simple process since we would only need to determine which specific arguments
to use with the CLI.

This is especially true for daemon processes which are used to start other processes, like the Docker daemon. We may
have specific requirements for the behavior of the network used by Docker containers. The Docker CLI provides
commands to create and manage the networks assigned to specific containers, so we would be able to change this aspect
of the program's behavior without stopping and restarting the whole Docker daemon.

\subsection{Configuration File Find and Replace}

Any software that uses simple configuration files is easy target for configuration file remediation. After scanning
the file to see if a requirement is met, we will likely be able to use a program such as sed to find the offending
configuration and modify it.

The steps involved to remediate this would be

\begin{itemize}
    \item Modify the configuration issue with sed, e.g. 'sed -i 's/foo/bar/' /var/?/conf
    \item Restart the program's process
\end{itemize}

\subsection{XML/JSON/YAML File Modification}

Programs that use a DSL or markup language to define their behavior could also be fixed with a find and replace action,
but the specific programs used would need to include a specific program to parse and modify the markup.

The steps involved with this type of configuration file would include

\begin{itemize}
    \item Read and parse configuration file
    \item Modify offending attribute
    \item Update configuration file
    \item Restart the program
\end{itemize}

Notice that this is similar to a basic configuration file find and replace, except the 'find and replace' action would
need to be done within a parser for the DSL.

\subsection{Manual Remediation}

Some issues are too complex or specific to the system to be resolved through automated means. These issues will require
an engineer to analyze the system to determine the best course of action.
