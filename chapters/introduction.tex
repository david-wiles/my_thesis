
\chapter{INTRODUCTION}

A glance at the offerings provided by Amazon Web Services shows that web hosting in 2022 is heavily based on
virtualization.  Virtual machines are cheaper for the provider and customer, more efficient, and easier to work with.
Entire fleets of cloud services can be created programmatically using tools such as Terraform or Chef. Additionally, after
the introduction of Docker in 2013 containers have quickly become the preferred way to test and deploy code due to their ease
of use.  Since there may be software from many different customers running on a single CPU concurrently, we may assume
that this would be a major security issue.

Thanks to the work of many talented developers working on KVM, LXC, and Docker, we can assume that there are a limited
number of critical vulnerabilities in the actual virtualization layer. The majority of issues arise from the misconfiguration
of the host machine or the virtualization itself.  This may include overly-permissive rules for programs or users,
exposed secrets, a lack of resource management, and other issues which will be discussed in this paper. Although by
default the attack surface for these services is very large, we can use tools to detect issues and automatically fix
them to ensure that host systems are secure from attacks from within the virtual machines.

\section{Complicance and Malware Scanning}
Compliance scanners exist which detect configuration issues as defined by a standard compiled by an organization
defining standards for system configurations. These standards may be required for certain organizations or governments
to ensure security, but can also provide a good starting point for a business or organization which does not have this
compliance regulated. OpenShift, a platform provided by Red Hat, does include a compliance scanner with the service.
This scanner will determine whether the system configuration meets the standards. Similarly, Docker, Inc. maintains
a project on GitHub that will automatically scan a Docker setup for any issues~\cite{docker_bench_security}.

These scanners can be combined with dynamic binary analyzers to determine if a program has been infected at runtime,
such as through a specially crafted input to a library such as log4j. Choi, Park, Eom, and Chung detail a system that
could be used to determine the presence of malware at runtime~\cite{ChoiYoung-Hyun2014Dbaf}. Despite these protections,
a system should always be built to withstand any attacks as much as possible through careful analysis of the system's
configuration.

\section{Common Attacks}

In general, an attacker may wish to achieve one of three goals when compromising a system. These goals are gaining
control of the system, reading sensitive data, or denying access to the system. Proper configuration of the system is
critical to preventing the attacker from achieving these goals, even if they can exploit one part of the system.

\subsection {InfoLeak}
A related issue to improper password storage is improperly secured sensitive data. Any data generated or
stored by a server should be considered sensitive since an attacker could infer a considerable amount from log files
or other seemingly innocuous files. A careless setting on these log files could expose customer information or
company secrets to an attacker, even if the attacker doesn't have root access to the machine.

Additionally, hardware issues such as timing attacks can also lead to an attacker gaining knowledge of a system. These
attacks are generally harder to achieve since the attack would need to involve multiple components, but can still be
mitigated by configuring the resources available to each VM or container

\subsection {Denial of Service}
Denial of service is likely the most common attack on cloud systems today. This may come in the form of ransomware,
general degradation of service, or a distributed denial-of-service attack. The most well-known denial-of-service attack
comes in the form of a network attack, where a server is flooded with requests to the point that service is degraded or
unavailable for actual users. While this kind of attack can be mitigated through a variety of methods, they aren't
relevant for this paper and so will not be discussed.

Other forms of denial of service attacks can come from malicious guests on a host. Consider a public cloud environment
that allows users full access to a guest VM. The malicious actor could attempt to gain control of many VMs on
a single machine and overuse the host's resources to degrade the performance of the other guests, especially if another
guest is the target. This technique is also used in side-channel attacks, mentioned previously under InfoLeak.

\subsection{Privilege Escalation}
Privilege escalation is the act of gaining more permissions than intended on a system. Gaining control of a system is
one of the most critical issues for a server machine since it could lead to any number of attacks. A malicious actor could
gain control in several, but the most well-known is through remote code execution. A bug in a program,
such as the recently discovered log4j vulnerability, can allow the attacker to execute an arbitrary code sequence
through a carefully-crafted input~\cite{MichaelCooney2021Lfni}. Once the attacker has this capability, they could
easily start a reverse shell with root permissions and steal information or demand a ransom to unencrypt data.
