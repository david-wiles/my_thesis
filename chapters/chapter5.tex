\chapter{KVM Host Considerations}

A machine used as a host for KVM virtual machines should be setup using its own set of guidelines to ensure that the
attack surface between VM's or a VM and the host is minimized.

\section {General Guidelines and Considerations}

Following the general guidelines for best practices when setting up a Linux server should be the first step when
preparing a machine to host guest VM's using KVM. However, there are some extra things to consider since the code on
the guest should not be trusted, especially in a cloud environment. Most of these considerations are centered around
the additional devices which may be used with a VM and IO access from the guest to the external network. KVM allows for
the guest's CPU instructions to be sent directly to the processor, if hardware virtualization is supported. This
drastically reduces the attack surface and allows us to focus on the components enabling communication between a guest
and any external device or network. We can use the guidelines outlined by Red Hat, Inc. as a good starting place for
determining the most secure system configuration~\cite{redhatcustomerservice}.

\subsection{The KVM Hypervisor}

Virtualization today relies heavily on KVM in the linux kernel. Most cloud providers will use this software in order
to create and manage virtual machines, although they may use proprietary software to manage the frontend of the
virtualization. If the CPU has hardware virtualization enabled, then KVM uses hardware virtualization to accomplish
the virtualization of an operating system. This has the benefit of separating the entire guest OS from the host OS
without the performance drawbacks of software virtualization. However, it is important to note that even with
harware acceleration KVM is less performant than linux containers, as shown by Chae, Lee, and Lee~\cite{ChaeMinSu2017Apco}.
Therefore, a KVM-based environment should be chosen for the security benefits whenever the performance drawback is
acceptable.

On a machine hosting VM's, special attention must be paid to the IO operations allowed by the guest machines since this
is the only part of the VM which must be processed by the host OS. Default settings may be acceptable depending on the
virtual machine monitor used but should still be verified by the administrator.

\subsection {Comparision to container-based Environments}

Use of a virtual machines for a shared-tenant environment like a public cloud is preferable to a container-based
environment. Albakri, Shanmugam, Samy, Idris, and Ahmed outline several risk assesment considerations for an
organization looking to move assets to a public cloud. They found that the responsibility for security in a cloud
computing environment is shifted to a the cloud provider in a public cloud, and an organization looking to shift
resources to this environment should take this into consideration when evaluating security risks~\cite{AlbakriSameerHasan2014Sraf}.
Due to this, a cloud provider looking to sell its resources should ensure that all possible risks of attacks between
VM's have been mitigated.

\subsection {Overview of issues}

\begin{center}
    \begin{tabular}{||p{0.3\textwidth} p{0.3\textwidth} p{0.3\textwidth}||}
        \hline
        Issue & Attack Type & Affected Components \\ [0.5ex]
        \hline\hline
        Insecure storage devices & Privilege escalation & File System \\
        \hline
        Unused external devices & Arbitrary code execution & Device drivers \\
        \hline
        Untrusted VM Images & InfoLeak, RCE & Network \\
        \hline
        Non-isolated guest networking & Infoleak & Networking \\ [0.5ex]
        \hline
    \end{tabular}
\end{center}


\section{Securing Storage Devices}

\subsection {Physical Storage Devices}
A guest VM with access to an entire storage device may be able to attack the host or other guests by rewriting volume
labels on the device. Unless the system requirements dictate that a guest should access an entire physical device, this
should be disallowed. There are a large number of possible file system types allowed with KVM, but the only one that
could cause issues on the host itself is access to an entire disk. If access to a physical drive is required, we should
ensure that it is done through a partition instead of the device. Additionally, we should make sure that the host
machine does not use file system labels to identify the file systems in the fstab file.

\subsection {Network Storage Devices}

Although a performance hit may be incurred, a storage device using the network may be a more sustainable and secure
solution if possible. This would allow for easier storage backups and failure tolerance. An extra consideration from
this perspective is to ensure that the data is encrypted while transferred and at rest and the encryption keys are
properly stored.

\subsection{Mitigation}

Under most circumstances, it is better to grant a guest access to a partition instead of an entire device. A partition
can be resized or moved as needed, something not possible with a single drive. Additionally, the storage device can be
shared between many guests and partitions will be cleared and reused as needed.

We can easily determine from the host whether the guest has an entire storage device or a partition. The fix for this
is to simply move the guest's data into a partition on the same disk.

\section{Disable Access to Physical Devices}

A similar issue to granting a guest an entire storage device is granting a guest access to any necessary hardware such
as USB devices. Depending on system requirements this may be necessary for a special system, but this would still
increase the attack surface significantly and should be avoided.

\begin{figure}[h]
    \begin{center}
        \includegraphics[width=1\textwidth]{./fig/KVM IO.drawio.png}
    \end{center}
    \caption{Device driver code required for external devices contributes significantly to the attack surface of a hypervisor.}
    \label{fig5}
\end{figure}

\subsection{Mitigation Approaches}

In a public cloud, there is no need to enable any external devices unless it is a business requirement. The vast
majority of cloud computing will only require access to block storage and the network. There exist entire virtual
machine monitors built around the concept of stripping the VMM down to only what is necessary for cloud computing,
whether is it in the form of short-lived server less functions, containers, or entire virtual machines which may run
for long periods of time. Firecracker, developed by Amazon for AWS Lambda, is probably the most relevant example of
this today. This VMM replaces QEMU and reduces the attack surface significantly by omitting any device drivers for
external devices.

\section{Use Secured VM Images}

\subsection{Image Storage}
VM images should be stored in a single, secure location. The images should not be accessible by any users other than the
ones who must read or modify them. Following the principle of least privilege is the best choice here since VM images
are a prime target from malicious code. Although VM images are a prime target, it is not any more complex to secure
these files than any other. As long as it is not publicly readable or writable and the relevant app armor profiles are
set up, the images can be considered secured. We can check this along with the other file permission checks.

The main concern with VM images is that the images are stored in alternate locations which may not have the same
security precautions as the default location. A cautious sysadmin will ensure that this does not happen, but the concern
is that a rogue user stores a malicious VM somewhere else on the system and creates a guest which may infect the host.
Properly following the rules of least privilege can prevent this from happening, and auditing the users as well as the
app armor profiles for relevant programs can stop users from attacking the system this way.

\subsection{Hardened VM Images}

Another easier solution than modifying an existing system would be to start from an OS image which is pre-configured to
be compliant with a certain standard. This may be the most efficient solution since the system administrator will not
need to modify the system in any way. However, we should consider that this will not be possible when working with an
existing system, and there may not be an image provided which meets the requirements for an organization. Therefore, a
general-purpose tool which can be used on any system to detect misconfiguration and fix trivial issues would be useful.

\section{Set Up libvirt Audits}

Detailed log files are essential to investigating errors which occur on production systems, but can also be helpful when
investigating security issues. As recommended by RedHat, usage of auvirt can help by recording all events that happen
during operation. In public cloud environments, this should be combined with logging of events relating to the user's
use of the system configuration. For example, any changes to networking should be logged, which could then be used to
analyze the VM's events to identify anomalies.

\section{Disable Networking Between Host and Guest VMs}

A guest VM should function as if it was an independent system, especially if the VM is in a public cloud. Unless the
requirements state that the VM guests should be able to communicate, there should be no network communication allowed
between the guests.

\section{VM Guest Security Considerations}

As with a KVM host, all procedures to secure a bare metal machine should be followed when configuration a guest VM.
Under most circumstances a guest will not function differently in a virtualized environment compared to a bare metal
one, so we should take care to set it up the same way.

\subsection{Differences Between VM and Bare Metal}

Any remote control of a machine should be done over encrypted channels only. A guest VM can only be accessed via remote
methods, unlike a bare metal host which can be accessed directly with a physical terminal. Similar to the configuration
of the host system, we can detect the method through which the user has connected to the guest and determine whether it
was secure. If the method was insecure, we must alert the user and give them the required steps to take to disable the
method and enable SSH.

\subsection{Storage Encryption}

Although a virtual machine's OS is separated from the host, the physical storage is still on the same drives as shared
by the hosts and other guest machines. To prevent sensitive data from leaking accidentally or maliciously, the guest's
data should be encrypted in a way which only that guest VM can read the data. Of course, an attacker with root access
on a machine would be able to read encrypted data on a physical disk since the key is stored on the host. However, if
this storage is in a separate location, such as a network drive, the data would be secure from any third party attacks
such as man-in-the-middle snooping.

When analyzing the storage configuration of the guest, we can determine whether the storage attached to the guest has
encryption set up. If not, we can automatically add an encryption format and generate a key for the user to then store
in a secure location.
