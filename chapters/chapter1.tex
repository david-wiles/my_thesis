\chapter{INTRODUCTION}

\section{Current Cloud Computing Landscape}

\subsection{Public Cloud and the Linux Kernel}
A quick glance at the offerings provided by Amazon Web Services shows that web hosting in 2022 is heavily based around
virtualization.  Virtual machines are cheaper for the provider and customer, more efficient, and easier to work with.
A new, identical virtual machine can be created programmatically using tools provided by Amazon. Additionally, after
the introduction of Docker in 2013 containers have quickly become the preferred way to test and deploy code due to their ease
of use.  Since there may be software from many different customers running on a single CPU concurrently, we may assume
that this would be a major security issue.

Thanks to the work by many talented developers working on KVM, LXC, and Docker, we can assume that there are a limited
number of critical vulnerabilities in the actual virtualization layer. The majority of issue arise from misconfiguration
of the host machine or the virtualization itself.  This may include overly-permissive rules for programs or users,
exposed secrets, a lack of resource management, and other issues which will be discussed in this paper. Although by
default the attack surface for these services is very large, we can use tools to detect issues and automatically fix
them to ensure that host systems are secure from attacks from within the virtual machines.

\subsection{Complicance and Malware Scanning}
Compliance scanners exist which detect configuration issues as defined by a standard compiled by an organization
defining standards for system configurations. These standards may be required for certain organizations or governments
to ensure security, but can also provide a good starting point for a business or organization which does not have this
compliance regulated. OpenShift, a platform provided by Red Hat, does include a compliance scanner with the service.
This scanner will determine whether the system configuration meets the standards.

These scanners can be combined with dynamic binary analyzers to determine if a program has been infected at runtime,
such as through a specially crafted input to a library such as log4j. Choi, Park, Eom, and Chung detail a system which
could be used to determine the presence of malware at runtime~\cite{ChoiYoung-Hyun2014Dbaf}. Despite these protections,
a system should always be built to withstand any attacks as much as possible through careful analysis of the system's
configuration.

\section{Common Attacks}

In general, an attacker may wish to achieve one of three goals when compromising a system. These goals are gaining
control of the system, reading sensitive data, or denying access to the system. Proper configuration of the system is
critical to preventing the attacker form achieving these goals, even if they are able to exploit one part of the system.
Gaining control of a system is the most critical issue for a server machine. This could be achieved by exploiting bugs
in programs leading to arbitrary code execution, but is probably more commonly caused by improperly hidden passwords.
A rogue user on a system can cause damage even if they don't have root access.

\subsection {InfoLeak}
A related issue to improperly stored or hidden passwords is improperly secured sensitive data. Any data generated or
stored by a server should be considered sensitive, since an attacker could infer a considerable amount from log files
or other seemingly innocuous files. Additionally, hardware issues such as timing attacks can also lead to an attacker
gaining knowledge of a system without leaving the data itself exposed.

\subsection {Denial of Service}
Denial of service is likely the most common attack on cloud systems today. This may come in the form of ransomware,
general degradation of service, or even a distributed denial of service attack.
