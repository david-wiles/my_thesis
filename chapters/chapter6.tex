\chapter{Conclusions and Future Work}
\section{On the Utility of Security Scanners and Audits}

Security in cloud environments is difficult to accomplish. There are countless attack surfaces when virtualization or
containerization is used and standards are constantly changing to adapt to new research or software. Although an
experienced administrator may have knowledge of the entire security best practices for their system, it is much more
thorough to maintain a program which can scan a system and automatically fix any major issues. Additionally, these
scanners can be run periodically to ensure that a system was not unintentionally modified or to ensure that new
standards have been met.

Although my program aims to focus on configuration of the operating system itself, similar programs could be created
to scan the configuration of a specific software. For example, it would be prudent to scan any database systems in
production for configuration errors to ensure that no unauthorized access is possible. We also see a form of security
scanner in the NPM package manager. This package manager will alert the user if any packages used in their program
contain known security vulnerabilities, and provide an easy way to upgrade those packages if possible.

\section{Future Work}

Tools such as this one may be built into the server programs they are associated with. A database server may give a
warning when the root user is set up, for example. This is a minimal example of what this program does. The primary
issue with this kind of tool is that the system's configuration may be dependent on the organization’s requirements.
However, we can make some general assumptions about the system and also allow for future extensibility.
