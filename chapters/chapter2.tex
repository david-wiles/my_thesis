\chapter{Canary Purpose, Design, and Usage}

This program is fairly simple and was designed to be extendable based on the needs of an individual or organization.
There exist similar security scanners like this, but this one was built to be generic for any Linux system and easy to
integrate with common provisioning frameworks.

\subsection{Purpose}

My goal in creating this software was to see if a system could be examined for common vulnerabilities programmatically.
Although there are many books written about setting up a system securely and many VM images can be found with best
practices already configured, I was unable to find a generic program which could assist in scanning a system for issues.
In light of this and driven by my curiosity in the importance of system configuration in security, I decided to start
on a simple program to scan systems for best practices in system configuration. This program was eventually called
'canary' and was expanded to include scanning for Docker and KVM issues, as well as allowing for user-configured tests
driven by shell files.

\subsection{Inspiration}

The inspiration for my system came from the package manager for node, 'npm'.  Whenever the user installs new software,
npm will automatically scan the packages for vulnerabilities and generate a report for the user. The software will also
automatically upgrade the affected packages if possible. I envisioned a similar program for linux machines which would
scan a system for misconfiguration and report to the user issues which were found. Then, if these issues could be easily
fixed programmatically, then the user could run a script to fix the issues. Of course, this does not guarantee safety
or even that the system would work, but it would provide a good starting point for an engineer who is setting up a system.

\subsection{Development and Iteration}

The first version of my program used the Docker and KVM API's directly to analyze the host and guest systems. I soon
realized that the same actions could be done through shell scripts with a higher degree of customization available to
the end user, with no affect on functionality. The final version uses a separate directory for analysis scripts, which
opens the possibility of having a separate 'registry' of canary scripts from which the user could download and upgrade
as needed.

\subsection{Design}

This program is a simple command-line program, written in Go. As with any other command-line program, the arguments are
first parsed to determine the specific action. The program will then determine the user's intent and scans the system
for the specified types of vulnerability. Once the scan is complete, if a fix is available then the user is prompted
to automatically fix the issue.

\begin{figure}[H]
    \begin{center}
        \includegraphics[width=0.5\textwidth]{./fig/Canary Flowchart.drawio.png}
    \end{center}
    \caption{The execution of Canary is essentially a single loop with runs bash files.}
    \label{fig1}
\end{figure}

\subsection{Usage}

Canary is a command-line program, so it naturally can be used by an administrator or engineer. However, it also could
be used effectively in a devops pipeline to ensure that security considerations are met before deployment.
