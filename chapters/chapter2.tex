\chapter{Canary Purpose, Design, and Usage}

This program is fairly simple and was designed to be extendable based on the needs of an individual or organization.
There exist similar security scanners but this one has been specifically created to be generic and extendable.

\subsection{Purpose}

My goal in creating this software was to see if a system could be examined for common vulnerabilities programmatically.
Although there are many books written about setting up a system securely and many VM images can be found with best
practices already configured, I was unable to find a generic program which could automatically fix a system's issues.
In light of this and driven by my curiosity in the importance of system configuration in security, I decided to start
on a simple program to scan systems for best practices in system configuration. When developing the system, I chose to
focus on container and virtualization technologies since that is where my focus had shifted from, but the program is
generic and would work for any type of automated vulnerability scanning and resolution.

\subsection{Inspiration}

A major inspiration for my system came from the package manager for node, 'npm'.  Whenever the user installs new software,
npm will automatically scan the packages for vulnerabilities and generate a report for the user. The software will also
automatically upgrade the affected packages if possible to resolve these issues. I envisioned a similar program for linux machines which would
scan a system for misconfiguration and report to the user issues which were found. Then, if these issues could be easily
fixed programmatically, then the user could run a script to fix the issues. Of course, this does not guarantee safety
or even that the system would work, but it would provide a good starting point for an engineer who is setting up a system.

\subsection{Development and Iteration}

The first version of my program used the Docker and KVM API's directly to analyze the host and guest systems. I soon
realized that the same actions could be done through shell scripts with a higher degree of customization available to
the end user with no effect on functionality. Both of these projects have an expressive CLI which allows the user to
perform any action or read information about any resource related to the program. This led me to believe that a better
approach would be to combine the utility of the existing CLI with my program.

The final version uses a separate directory for analysis scripts, which opens the possibility of having a separate
'registry' of canary scripts from which the user could download as needed. Consider an engineer who is responsibile
for setting up a new system composed of multiple technologies deployed on separate machines. This engineer would likely
be using a tool which automates the process of provisioning and deploying the system. In order to ensure that each
individual system is set up securely each time, they could place canary as part of a pre-deployment step to verify that
there are no configuration issues. This hypothetical step would just need to install canary and download an existing
set of scripts for the corresponding technology stack.

\subsection{Design}

This program is a simple command-line program, written in Go. As with any other command-line program, the arguments are
first parsed to determine the specific action. The program will then determine the user's intent and scans the system
for the specified types of vulnerability. Once the scan is complete, if a fix is available then the user is prompted
to automatically fix the issue.

\begin{figure}[H]
    \begin{center}
        \includegraphics[width=0.5\textwidth]{./fig/Canary Flowchart.drawio.png}
    \end{center}
    \caption{The execution of Canary is essentially a single loop with runs bash files.}
    \label{fig1}
\end{figure}

\subsection{Usage}

Canary is a command-line program, so it naturally can be used by an administrator or engineer. However, it also could
be used effectively in a devops pipeline to ensure that security considerations are met before deployment.
