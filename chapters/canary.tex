\chapter{Canary: A CLI for Security}

Canary is a generic security scanner and resolver meant to be extendable through the creation of independent packages.

\section{Canary Background}

\subsection{Purpose}

My goal in creating this software was to see if a system could be examined for common vulnerabilities programmatically.
Although there are many books written about setting up a system securely and many VM images can be found with best
practices already configured there is not a well-known all-purpose program that could be used to securely set up a system.
Canary was created in an attempt to demonstrate that this void could be filled relatively simply using a combination of
a command-line program and remote repository.

\subsection{Inspiration}

The initial inspiration for Canary came from the package manager for node, 'npm'.  Whenever the user installs new software,
npm will automatically scan the packages for vulnerabilities and generate a report for the user. The software will also
automatically upgrade the affected packages if possible to resolve these issues. A similar program for Linux machines could
scan a system for misconfiguration and report to the user issues which were found. If these issues could be easily
fixed programmatically then the user could run a script to fix the issues. Although this does not guarantee the safety
or even that the system would work, it does provide a good starting point for an engineer who is setting up a system.

\section{Canary Implementation}

Canary is a CLI but a related remote repository is crucial for the program's viability as a security solution.

\subsection{Canary Usage}

There are four commands included with this first version of Canary.

\begin{center}
\begin{tabular}{||p{0.2\textwidth} p{0.7\textwidth}||}
    \hline
    Command & Description \\ [0.5ex]
    \hline\hline
    Install & Installs packages from a remote repository  \\
    \hline
    Upgrade & Upgrades packages to their latest version \\
    \hline
    Remove & Removes packages from local storage \\
    \hline
    Check & Uses packages to check for issues and resolve them \\ [0.5ex]
    \hline
\end{tabular}
\end{center}

With each command the option '--local' can be passed to specify the local storage for packages, otherwise the
directory '\~/.canary' will be used. The remote repository can also be specified on the command line if the default is
not desired.

On the check command, it is possible to specify a package not from the remote repository by using the flag
--directory to specify the root directory of the custom package. This command also supports '--auto-fix' to fix
issues without prompting the user and '--scan-only' to report issues without prompting the user to fix them.

\begin{figure}[h]
    \begin{center}
        \includegraphics[width=1\textwidth]{./fig/Canary Flowchart.drawio.png}
    \end{center}
    \caption{High-level overview of Canary program execution.}
    \label{fig1}
\end{figure}

\subsection{Design}

Canary is a command-line program written in Scala. It contains a custom argument parser which then initializes specific
commands based on the user input. The only runtime dependencies of the project are Softwaremill's STTP for an HTTP client
and the Compress library Apache Commons to extract gzipped archives downloaded from the remote Canary repository.

The install command checks the remote repository for the specified packages and downloads them into a local directory if
they exist. It then verifies the checksum from the remote repository and extracts the archive into the local folder so
the package can be used.

Upgrade performs a similar action to install, but only acts on packages that already exist locally. This is to prevent
unintentional actions by the user.

Remove will check that a package exists and remove all files related to it in the local Canary root directory.

The check command performs the actual software scanning and issue resolution. The check command will use the specified packages and
run each 'analyze' file to determine if there are security issues by checking the exit status of the program. If the
exit status was not 0, then Canary will prompt the user and run any provided solution file to resolve the issues
identified.

\subsection{Canary Packages}

A Canary package is a directory containing tasks. These tasks are also directories, which contain shell scripts and a
configuration file, which should be called 'config'. This file will specify the name of any other scripts to run and
include a description of the issue.

\subsection{The Canary Repository}

Crucial to the viability of Canary as a way to automatically scan issues is a remote repository to maintain packages.
Successful software today is dependent on a seamless way to distribute and extend itself, such as with npm. Downloading
and executing software is always risky, so users can be sure that the packages they download are trusted and have been
verified to reduce this risk.

This also makes it more simple for third parties to add security checks for their own programs. An organization that
wants to release a package on Canary could submit their package to be stored on the Canary repository instead of
trying to distribute their own package and security scanner. Approaching security this way allows the project maintainers
to update security best practices as soon as they change and immediately distribute the changes to all users of Canary.

