
\chapter{Conclusions and Future Work}

Automated tools like Canary can save an engineer's time and allow them to focus on issues that require attention
instead of tasks that can be automated once an issue has been detected. They can also allow for an organization to
run regular audits on itself to demonstrate that they have set up a system using the industry's standards to
mitigate attacks.

In an environment where vulnerabilities are more likely to come from an overlooked setting on the part of
an engineer instead of a bug in the application in question, an automated scanner to prevent these issues would be welcome.
Security in cloud environments is difficult to accomplish. The attack surface can be large even with modern tech stacks
and standards are constantly changing to adapt to new research or software. With this in mind, there is a place for
automated scanners and configuration tools to improve the workflow and security practices of many organizations.

\section{Future Work}

\subsection{Canary Packages}

Canary functions as the engine which runs security audits but does not come with a thorough set of tests included. An
important next step for Canary is to complete a production-ready set of tests for an application. Although it would be
best for these packages to be maintained by the same organization which maintains the application itself, it is more
likely that these tests would be created by open-source contributors. It is critical for the application's viability
as a security solution to have a full suite of tests for many commonly-used applications or technologies.

\subsection{Canary Repository Interace}

Open-source repositories such as NPM have succeeded due to their ease of use as well as the ease with which contributors
can submit their packages to the repository for hosting. The Canary repository is currently a simple file server,
but a full user interface to allow package maintainers to submit packages for review would allow for the repository to function
as a viable way for users to share their packages for security audits.

